\documentclass[a4paper, 11pt]{article}

% DEFINIËREN CODERING
\usepackage[utf8]{inputenc}

% DEFINIËREN BLAD
\usepackage[
top=20mm,
bottom=20mm,
left=20mm,
right=20mm,
heightrounded,
]{geometry}
\usepackage{lastpage}
\usepackage{fancyhdr}
%\pagestyle{fancy}
\fancyhead[L]{}
\fancyhead[C]{}
\fancyhead[R]{}
\fancyfoot[L]{}
\fancyfoot[C]{\thepage}
\fancyfoot[R]{}
\renewcommand{\headrulewidth}{0pt}
\renewcommand{\footrulewidth}{.4pt}
\setlength{\headheight}{13.6pt}
\setlength{\parindent}{1cm}

% MEDIA
% Tabellen
\usepackage{array}
\newcolumntype{L}[1]{>{\raggedright\let\newline\\\arraybackslash\hspace{0pt}}m{#1}}
\newcolumntype{C}[1]{>{\centering\let\newline\\\arraybackslash\hspace{0pt}}m{#1}}
\newcolumntype{R}[1]{<{\raggedleft\let\newline\\\arraybackslash\hspace{0pt}}m{#1}}
\usepackage{stackengine}
\newcommand\xrowht[2][0]{\addstackgap[.5\dimexpr#2\relax]{\vphantom{#1}}}
\usepackage{tabularx}
\usepackage{multirow}
\usepackage{hhline}
\usepackage{float}
% Afbeeldingen
\usepackage{graphicx}
% Figuren
\usepackage{tikz}

% WISKUNDE
% Basis
\usepackage{amsmath}
\usepackage{amssymb}
\usepackage{amsthm}
% Aanpassing
\usepackage{dsfont}   % Voor de natuurlijke, gehele, rationele, reële, complexe getallen => ``$\mathfb{}$''
\usepackage{mathdots} % Voor betere puntjes
\usepackage{esvect}   % Voor betere vectorpijltjes => ``$\vv{}$''
\usepackage{textcomp} % Nodig om gensymb correct te laten werken
\usepackage{gensymb}  % Invoegen van enkele symbolen => ``\degree ; \celcius ; \perthousand ; \ohm ; \micro''
\usepackage{cancel}   % Voor het beter doorstrepen van zaken => ``\cancel ; \bcancel ; \xcancel''
\usepackage{esint}    % Meervoudige kringintegralen
\usepackage{hyperref}

\newcommand{\naar}{\,$\rightarrow$\,}
\newcommand{\ppp}{\ldots}
\renewcommand{\empty}{$\varnothing$}
\newcommand{\of}{$\cup$}
\newcommand{\<}{\scriptsize\textless\normalsize}
\renewcommand{\>}{\scriptsize\textgreater\normalsize}

% Voorblad
\title{Programmeerproject 2: Besturingssysteem Modeltreinen\\ Handleiding voor Eindgebruikers}
\author{Jonas Br\"ull, 0587194\\ Jonas.Simon.E.Brull@vub.be\\}
\date{Academiejaar 2024-2025\\Vrije Universiteit Brussel}


% BEGIN DOCUMENT ==================================================================================
\begin{document}

\pagenumbering{gobble}
\maketitle
\newpage

\pagenumbering{gobble}
\tableofcontents
\newpage

\pagestyle{fancy}
\setcounter{page}{1}
\pagenumbering{arabic}

% =================================================================================================
\section{Introductie} % ===========================================================================
Dit document beschrijft de handleiding van mijn implementatie van een Modeltrein Software Toepassing in de programmeertaal Racket. Het is deel van het opleidingsonderdeel ``Programmeerproject 2''.\\\\
*** In de huidige staat van mijn project, is het nog niet mogelijk voor een eindgebruiker, om de applicatie te bedienen. ***

\newpage
% =================================================================================================
\section{Nodige Software en Bestanden} % =======================================================================
Alle benodigde bestanden zijn gebundeld in het zip-bestand ``besturingssysteem-modeltreinen.zip'' of te vinden op de GitHub repository van het project. De applicatie is geschreven in de programmeertaal Racket, en kan worden uitgevoerd met DrRacket.\\\\
De applicatie is afhankelijk van de volgende Racket packages:
\begin{itemize}
  \item \textbf{racket/gui}:  Voor de gui elementen.
  \item \textbf{racket/date}: Voor het werken met datum \& tijd tijdens het loggen van gebeurtenissen. 
  \item \textbf{racket/exn}:  Voor het afhandelen van uitzonderingen.
  \item \textbf{try-catch}:   Voor het afhandelen van fouten tijdens het opstarten. 
\end{itemize}
Buiten try-catch, zijn deze packages standaard geïnstalleerd in DrRacket. DrRacket zal normaal gezien zelf voorstellen om try-catch te installeren.\\

\subsection{Installatie DrRacket} %----------------------------------------------------------------
DrRacket is te vinden op de website: \url{https://racket-lang.org/} onder "Download". De standaardwaarden bij het installeren van DrRacket zijn voldoende voor dit project.\\\\

\subsection{Opstarten} %---------------------------------------------------------------------------
Het opstarten van de applicaties gebeurt via het openen van volgende bestanden in DrRacket:
\begin{itemize}
  \item \textbf{Infrabel}: INFRABEL.rkt
  \item \textbf{Providers}: NMBS.rkt, andere geconfigureerde providers.
\end{itemize}
De andere bestanden bevatten de applicatie logica en hoeven niet geopend te worden.\\\\
\begin{figure}[h]
	\begin{center}
		\includegraphics[scale=.5]{Bestanden/besturingssysteem-modeltreinen-tldir.png}
		\caption{Top Level Directory van het project}
	\end{center}
\end{figure}

\newpage
% =================================================================================================
\section{Infrabel} % ==============================================================================
\subsection{Opstarten} %---------------------------------------------------------------------------
Om de Infrabel applicatie te starten, moet het bestand INFRABEL.rkt geopend worden in DrRacket. Hierna moet je op het groene pijltje (start-knop) rechts-boven drukken. Het programma zal opstarten en de "Infrabel startup manager" zal openen.
\begin{figure}[h]
	\begin{center}
		\includegraphics[scale=.5]{Bestanden/infrabel-rkt.png}
		\caption{Opstarten van Infrabel via DrRacket}
	\end{center}
\end{figure}
\\In de Startup Manager kan je de instellingen aanpassen:
\begin{itemize}
  \item \textbf{Railway Architecture}: Infrabel kan verbinding maken met zowel simulators, als hardware opstellingen. Hier kun je aanpassen met welke opstelling je wilt werken. Let op: voor hardware, moet je eerst verbinding maken met de Z21 Command \& Control centrale van de harware.
  \item \textbf{Simulator version}: Hier kan je de versie van de simulator kiezen. Deze optie wordt uitgeschakeld als je voor hardware koos als architecture.
  \item \textbf{Hostname}: Standaard zal de infrabel server op localhost draaien, maar als je Infrabel op een Raspberry Pi wilt draaien, kun je hier het IP-address ingeven.
  \item \textbf{Port}: Hier kan je de poort van de Infrabel server kiezen. Standaard poort is 2020.
  \item \textbf{Start with Control Panel}: Indien je een Controle Paneel (GUI) wenst om Infrabel aan te sturen (bijvoorbeeld om fouten op te lossen), kun je deze optie aanvinken.
\end{itemize}
De standaard instellingen zijn voldoende om Infrabel te laten werken en gebruik je best ook als je de software voor de eerste keer gebruikt.\\
Na het aanpassen van de instellingen, kun je op de knop "Start" drukken. Infrabel zal nu opstarten en verbinding maken met de simulator of hardware.\\
Moest er een fout optreden tijdens het opstarten, kun je deze terugvinden in het "Log Paneel", naast de instellingen. Hier zal ook de log van de opstart verschijnen.\\
\begin{figure}[h]
	\begin{center}
		\includegraphics[scale=.5]{Bestanden/infrabel-startup-rkt.png}
		\caption{Opstarten van Infrabel via de Startup Manager}
	\end{center}
\end{figure}

\newpage

\begin{figure}[h]
	\begin{center}
		\includegraphics[scale=.35]{Bestanden/infrabel-gui-rkt.png}
		\caption{Infrabel controle paneel}
	\end{center}
\end{figure}


\newpage
% =================================================================================================
\section{Providers} % =============================================================================

% =================================================================================================
\label{lastpage}
\end{document}
