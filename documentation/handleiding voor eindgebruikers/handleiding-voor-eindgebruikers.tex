\documentclass[a4paper, 11pt]{article}

% DEFINIËREN CODERING
\usepackage[utf8]{inputenc}

% DEFINIËREN BLAD
\usepackage[
top=20mm,
bottom=20mm,
left=20mm,
right=20mm,
heightrounded,
]{geometry}
\usepackage{lastpage}
\usepackage{fancyhdr}
%\pagestyle{fancy}
\fancyhead[L]{}
\fancyhead[C]{}
\fancyhead[R]{}
\fancyfoot[L]{}
\fancyfoot[C]{\thepage}
\fancyfoot[R]{}
\renewcommand{\headrulewidth}{0pt}
\renewcommand{\footrulewidth}{.4pt}
\setlength{\headheight}{13.6pt}
\setlength{\parindent}{1cm}

% MEDIA
% Tabellen
\usepackage{array}
\newcolumntype{L}[1]{>{\raggedright\let\newline\\\arraybackslash\hspace{0pt}}m{#1}}
\newcolumntype{C}[1]{>{\centering\let\newline\\\arraybackslash\hspace{0pt}}m{#1}}
\newcolumntype{R}[1]{<{\raggedleft\let\newline\\\arraybackslash\hspace{0pt}}m{#1}}
\usepackage{stackengine}
\newcommand\xrowht[2][0]{\addstackgap[.5\dimexpr#2\relax]{\vphantom{#1}}}
\usepackage{tabularx}
\usepackage{multirow}
\usepackage{hhline}
\usepackage{float}
% Afbeeldingen
\usepackage{graphicx}
% Figuren
\usepackage{tikz}

% WISKUNDE
% Basis
\usepackage{amsmath}
\usepackage{amssymb}
\usepackage{amsthm}
% Aanpassing
\usepackage{dsfont}   % Voor de natuurlijke, gehele, rationele, reële, complexe getallen => ``$\mathfb{}$''
\usepackage{mathdots} % Voor betere puntjes
\usepackage{esvect}   % Voor betere vectorpijltjes => ``$\vv{}$''
\usepackage{textcomp} % Nodig om gensymb correct te laten werken
\usepackage{gensymb}  % Invoegen van enkele symbolen => ``\degree ; \celcius ; \perthousand ; \ohm ; \micro''
\usepackage{cancel}   % Voor het beter doorstrepen van zaken => ``\cancel ; \bcancel ; \xcancel''
\usepackage{esint}    % Meervoudige kringintegralen

\newcommand{\naar}{\,$\rightarrow$\,}
\newcommand{\ppp}{\ldots}
\renewcommand{\empty}{$\varnothing$}
\newcommand{\of}{$\cup$}
\newcommand{\<}{\scriptsize\textless\normalsize}
\renewcommand{\>}{\scriptsize\textgreater\normalsize}

% Voorblad
\title{Programmeerproject 2:\\ Handleiding Voor eindgebruikers}
\date{Academiejaar 2023-2024}
\author{Jonas Br\"ull, 0587194\\ (Jonas.Simon.E.Brull@vub.be)\\\\ 2ba Computerwetenschappen}

% BEGIN DOCUMENT ===============================================================
\begin{document}

\pagenumbering{gobble}
\maketitle
\newpage

\pagenumbering{gobble}
\tableofcontents
\newpage

\pagestyle{fancy}
\setcounter{page}{1}
\pagenumbering{arabic}

% ==============================================================================
\section{Introductie} % ========================================================
Dit document beschrijft de handleiding van mijn implementatie van een Modeltrein Software Toepassing in de programmeertaal Racket.\\
Het is een deel van het opleidingsonderdeel ``Programmeerproject 2''.\\
Concreet wordt in dit project verwacht om een controlesysteem te ontwikkelen dat modeltreinen kan aansturen vanaf een computer.

\section{Opstarten}
Het opstarten van het modeltreintjes project gebeurt voorlopig via het bestand \texttt{tcp/main.rkt}. Als je dit bestand runt in DrRacket, zullen 2 schermen openen, een \emph{GUI Simulator} en een \emph{GUI NMBS}. Via de laatste is het mogelijk om de simulator te besturen zoals hieronder uitgelegd. De GUI bestaat uit verschillende tabs, deze elk voor een eigen soort commando's instaat.

\section{SWITCHES-tab}
In deze tab kun je de wissels besturen. Elke wissel is aangeduid met diens naam, diens huidige stand en twee knoppen om diens stand te veranderen.

\section{CROSSINGS-tab}
In deze tab kun je de overwegen besturen. Elke overweg is aangeduid met diens naam, diens huidige stand en twee knoppen om diens stand te veranderen.

\section{LIGHTS-tab}
In deze tab kun je de lichten besturen. Elk licht is aangeduid met diens naam, diens huidige signaal en acht knoppen om diens signaal te veranderen.

% ==============================================================================
\label{lastpage}
\end{document}
